\documentclass[11pt]{article}
% this package is for page size
% \usepackage[legalpaper]{geometry}
% \usepackage[landscape, legalpaper]{geometry}
% geometry package can also change margins
% \usepackage[letter, left = 0.5 in, right = 0.5in, top = 0.5 in, bottom = 0.5 in]{geometry}
% margin attribute sets all four margins if you want to save time

% for this package geometry, overleaf seems much more permissive in its acceptance
% of attributes than the command line, which would simply be:
% pdflatex <filename.tex>
\usepackage[letterpaper, margin=2in]{geometry}
\usepackage[utf8]{inputenc}
\usepackage{lipsum}
\usepackage{comment}

\title{Document Title}

\author{Paul V. Miller\thanks{This is a footnote with an asterisk}\\Duquesne University \and Second Author\thanks{email: author@upitt.edu}\\University of Pittsburgh}

% leave date empty and no date will appear
% \date{}
% today's date
\date{\today}

\begin{document}

% title is a SET of fields defined above
\maketitle

% you can put an abstract here
\begin{abstract}
    \lipsum[7]
\end{abstract}

% This is not very elegant
Keywords: LaTeX, editor, practice

More text

\iffalse
A lot of text here will be ignored
\fi

% or do an abstract this way
% \abstract{This is the abstract of this essay.}

\section{Introduction}

\lipsum[1-2]

\begin{comment}
This is another large comment
\end{comment}

\end{document}
