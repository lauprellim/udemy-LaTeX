\documentclass{article}
% for inserting images
\usepackage{graphicx}

% this package gives more font face variation
\usepackage{ulem}

\title{Font Style Practice}
\author{Paul V. Miller}
\date{June 2025}

\begin{document}

\maketitle

\section{Introduction}

If we want to underline a word or sentence, you can use the command \underline{underline}. On the other hand, we can we might want to use boldface, which can be done by invoking command \textbf{textbf}. A third face change is italic, which you do \textit{textit for italics}.

You can also use \emph{emph} for italics, but not in all document packages. textit is more universal, according to Paulo.

Now we are using features from the ulem package.
\uline{The uline command} is equivalent to the command {underline}. By the way emph is now screwed up above. The ulem underline is a bit more separated from text than \underline{underline}. The uuline command can do \uuline{double underline}. This is silly: a wavy underline can be obtained \uwave{by using uwave}.

The ulem package also has \sout{strikethrough}, which could be very useful in many instances...

``Censorship'' (note ``quotation marks'') can be achieved by using xout: \xout{This is slacked out}. There are other examples: dashed line: \dashuline{underline with dash}. Yet another example: \dotuline{dotted underline}.

\clearpage

\noindent
{\tiny this is the smallest size} \\
{\scriptsize this is scriptsize} \\
{\footnotesize this is footnotesize} \\
{\small this is a small size} \\
{\normalsize This is the normal font size}
{\large this is large size} \\
{\Large this is Large size} \\
{\LARGE this is LARGE size} \\
{\huge this is huge size} \\
{\Huge this is Huge size} \\

Return to the regular size. You can also write
\normalsize normalsize.

\end{document}
